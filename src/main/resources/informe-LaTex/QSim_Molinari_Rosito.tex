\documentclass[10pt,a4paper,draft]{book}
\usepackage{Qsim}

\author{\susi \and \tati}
\title{\qsim: Simulador de arquitecturas Q}


\pagestyle	{fancy}
\begin{document}
\maketitle
\tableofcontents


%\chapter{Introducción}

\begin{abstract}


Una de las primeras asignaturas que debe recorrer un estudiante de la \tpi\ es \textbf{\orga}. En esta materia los estudiantes descubren los componentes funcionales que conforman un sistema de cómputos, con el fin de comprender un modelo de ejecución de programas que está presente hoy en día en la mayoría de las computadoras personales.
Este trabajo...
\end{abstract}
\part{Contexto}
%%%%%%%%%%%%%%%%%%%%%%%%%%%%%%%%%%%%%%%%%%%%%%%%%%%%%%%%%%%
\section{Sobre la materia \orga (susi)}

%%%%%%%%%%%%%%%%%%%%%%%%%%%%%%%%%%%%%%%%%%%%%%%%%%%%%%%%%%%
\section{Conceptos importantes}

\subsection{Enfoque de Von Neumann}

El matemático John Von Neumann en el año 1945 que por ese entonces se encontraba colaborando en el proyecto  ENIAC, se comenzó a interesar por la problemática que significaba la la necesidad de reconfigurar la máquina para cada nueva tarea y tan sólo cuatro años más tarde propone y desarrolla una solución a este problema la cual consistentía en poner la información sobre las operaciones a realizar en la misma memoria utilizada para los datos escribiendola en código binario al igual que los datos.

Este enfoque generaliza la organización de las computadoras distinguiendo en tres partes interconectadas: La CPU (con la unidad aritmético-lógica o ALU y la unidad de control) la memoria, y un módulo de entrada/salida. La interconexión es llevada a cabo por un un bus de sistema que proporciona un medio de transporte de los datos entre las distintas partes. \\

Con la propuesta de este modelo Von Neumann incorpora el concepto de \textbf{programa almacenado}\ojo{ contiene un conjunto de instrucciones que podían ser almacenadas en memoria, o sea, un programa que detalla la computación del mismo}{no entiendo esta frase. la puse yo? Si mara, jajaja que hacemos? (tati)}. Con esta idea, el programa se codifica de cierta manera para que pueda ser almacenado en memoria principal y posteriormente ejecutado quizás múltiples veces. De esta manera, la lógica del programa puede ser ''recordada'' y el programa toma un valor mayor, a diferencia de lo que ocurría hasta entonces, donde el programa se reflejaba en un conjunto de configuraciones de cables aplicadas a los equipos. Esto implica una separación entre el mecanismo de ejecución (el hadware) y la lógica de computo o instrucciones (el software).

Este tipo del diseño que permite un programa almacenado también da la posibilidad de que la ejecución de las instrucciones modifique el código máquina del mismo u otro programa. Por ejemplo un programa podría modificar o incrementar las referencias a las direcciones de memoria que tenga en algunas instrucciones y luego volver a ejecutar dichas instrucciones con el fin de procesar celdas diferentes de memoria cada vez. Esta característica es potente pero presenta un alto riesgo pues las modificaciones en los programas podía ser algo perjudicial, por accidente o por diseño.

\subsection{Organización de la computadora}

La CPU (Unidad Central de Procesamiento del inglés: Central Processing Unit), es el componente principal y el encargado ejecutar los programas y procesar los datos. La CPU contiene otros componentes de importancia tales como la \textbf{Unidad de Control}, el contador de programa (PC o \PC), el registro de instrucción (IR - \textit{Instruction Register}), otros registros de uso específico y general y la Unidad Aritmético-Lógica (ALU).\\

La \UC\ dirige el ciclo de ejecución de cada instrucción, pidiendo la lectura de celdas de memoria donde esta alojada, decodificándola  y ejecutándola luego en colaboración con los otros componentes del sistema: si es una operación lógica o aritmética le ordena a la ALU su ejecución, si es de movimiento de datos colabora con la memoria ó el módulo de Entrada/Salida.\\

El contador de programa (en inglés \PC\ o PC) es un registro que indica la posición de memoria donde estará la siguiente instrucción que debe ejecutarse. Luego de completar el ciclo de ejecución de una instrucción, el PC se incrementa en función de la cantidad de celdas que ocupa el código máquina de esta.\\

El registro de instrucción contiene el código máquina de la instrucción actual una vez que la misma es leída de memoria para luego decodificarla y ejecutarla.\\

El diseño de cada arquitectura ofrece un conjunto diferente de registros de uso general para ser usados en los programas. Estos registros son elementos de memoria de alta velocidad y poca capacidad que pueden ser utilizados como variables en los programas. Es importante marcar que pueden guardar tanto datos como direcciones de memoria.\\

La \ALU, recibe su nombre de las siglas en inglés de \textit{Arithmetic and Logic Unit}. La ALU es un circuito digital que lleva a cabo operaciones aritméticas (suma, resta, multiplicación, división) y las operaciones lógicas como la negación, disyunción, conjunción, etc, entre dos cadenas binarias que son interpretadas como números o valores lógicos.\\

La memoria es un conjunto de celdas numeradas. La numeración de cada celda la identifica inequívocamente por lo cual a esta numeración se le llama dirección. En cada celda de la memoria se pueden almacenar datos o instrucciones en forma de cadenas binarias y este contenido puede leerse y modificarse. En la memoria es donde se alojan los programas que luego serán ejecutados.\\

El bus de sistema es el encargado de transferir los datos entre los componentes de la computadora.La unidad de control al pedir un contenido de una dirección de memoria lo hace a través del bus, y similarmente mismo cuando desea escribir en memoria.

\subsection{Ejecución de un programa}

La función de una computadora es la ejecución de programas. Los programas se encuentran almacenados en memoria y consisten en una sucesión de instrucciones que posee orden. La CPU es quien se encarga de ejecutar dichas instrucciones a través de un ciclo denominado ciclo instrucciones. Para ser almacenadas en memoria, las instrucciones deben codificarse en cadenas binarias (secuencias de ceros y unos) que no son legibles para las personas pero que la UC puede interpretar y traducir en acciones. Por eso para saber de qué instrucción se trata, y cuales son los valores o celdas de memoria que debería consultar o usar, la cpu utiliza la UC, que tomando bit por bit interpreta la cadena binaria y verificando los códigos únicos de cada instrucción o modos de direccionamiento, sabe qué instrucción y con qué valores debería ejecutar. La ejecución de instrucciones se divide en tres etapas importantes, fech - decode - execute.

Al principio de cada ciclo de ejecución, durante el fech de instrucción la CPU busca una instrucción que se encuentra en alguna parte de la memoria. Para saber donde esta dicha instrucción la CPU contiene un registro llamado contador de programa (PC), que tiene la dirección de la próxima instrucción a buscar. La CPU a través del bus lee la instrucción, y luego incrementa el valor contenido en PC; así podrá buscar la siguiente instrucción en la secuencia luego de terminar con la actual. La instrucción leida que está en la forma de cadena binaria se carga dentro de otro registro de la CPU, llamado registro de instrucción (IR).

Durante la decodificacion la UC determina que significa la cadena binaria.

Finalmente al saber de qué instrucción se trata la CPU ejecuta la instrucción, es decir, realiza lo que la instrucción dice que debe hacer con sus argumentos, modificando la memoria o los registros como resultado final y el ciclo vuelve a comenzar hasta que el programa termine.

%%%%%%%%%%%%%%%%%%%%%%%%%%%%%%%%%%%%%%%%%%%%%%%%%%%%%%%%%%%
\section{Arquitecturas Q}

%%%%%%%%%%%%%%%%%%%%%%%%%%%%%%%%%%%%%%%%%%%%%%%%%%%%%%%%%%%
\subsection{Características generales} \label{caracteristicasQ}

\begin{itemize}
\item Tiene \textbf{8 registros} de uso general de 16 bits:\textbf{ R0..R7}.

\item La memoria tiene \textbf{direcciones de 16 bits}.

\item Tiene un contador de programas (\textbf{program counter}) de 16 bits.

\item \textbf{Stack Pointer} de 16 bits. Comienza en la direccion FFEF.

\item Flags: \textbf{Z,N,C,V}(Zero, Negative, Carry, Overflow).

\item Todas las instrucciones alteran los flags excepto \textbf{MOV, CALL, RET, JMP, Jxx}.

\item De las instrucciones que alteran los Flags, todas dejan C y V en 0 a excepcion de \textbf{ADD, SUB y CMP}.

\item Las instrucciones que alteran Z y N:\textbf{ ADD, SUB, CMP, DIV, MUL, AND, OR, NOT}. Las 3 primeras ademas calcula C y V.

\end{itemize}

\textbf{Nota1:} El caracter \% denota el cociente de la división entera.\\

\textbf{Nota2:} El resultado de la operación \textbf{MUL} ocupa 32 bits, almacenándose los 16 bits menos significativos en el operando destino y los 16 bits mas significativos en el registro \textbf{R7}.\\

\textbf{Nota3:} Donde \textbf{rrr} es una codificación (en 3 bits) del número de registro.\\

\textbf{Nota4:} Con este formato, los campos \code{\dest{}} y \code{\src{}} contienen valores constantes (si el modo respectivo es \textit{inmediato}), contienen direcciones de memoria principal (si el modo es \textit{directo}), o no existen (si el modo respectivo es \textit{registro}). \\

\textbf{Nota5:} Saltos condicionales tiene como formato codop - desplazamiento donde los primeros 4 bits del campo codop corresponde a la cadea 1111. Las instrucciones en este formato son de la forma Jxx(salto relativo condicional).Si al evaluar la condicion de salto en los Flags el resultado es 1, el efecto es incrementar al PC con el valor de los 8 bits de desplazamiento, representado en complemento a 2 de 8 bits. En caso contrario la instruccion no hace nada. \\

%%%%%%%%%%%%%%%%%%%%%%%%%%%%%%%%%%%%%%%%%%%%%%%%%%%%%%%%%%%
\subsection{Modos de direccionamiento}

\begin{enumerate}

\item \textbf{Inmediato}
Codigo de modo de direccionamiento: 000000
Este modo de direccionamiento es un valor que será utilizado como modo origen pero nunca como modo destino ya que en el no pueden guardarse datos.

Ejemplos:
\textbf{0x0000} representa el modo de direccionamiento inmediato cuyo valor es cero.
\textbf{0x000F} representa el modo de direccionamiento inmediato cuyo valor es 15.

\item \textbf{Directo}
Codigo de modo de direccionamiento: 001000
Este modo de direccionamiento representa una direccion de memoria o de puertos que será utilizado como modo origen pero nunca como modo destino ya que en el no pueden guardarse datos.

Ejemplos:
\textbf{[0x0000]} representa el modo de direccionamiento directo cuyo valor se encuentra en la celda de memoria que identifica uniquevocamente a la dirección \textbf{0x0000}, es decir 0.
\textbf{[0x000F]} representa el modo de direccionamiento directo cuyo valor se encuentra en la celda de memoria que identifica uniquevocamente a la dirección \textbf{0x000F}, es decir 15.

\item \textbf{Indirecto}
Codigo de modo de direccionamiento: 011000
Este modo de direccionamiento representa una direccion de memoria o de puertos que será utilizado como modo origen pero nunca como modo destino ya que en el no pueden guardarse datos.

Ejemplos:
\textbf{[[0x0000]]} representa el modo de direccionamiento indirecto cuyo valor se encuentra en la celda de memoria cuya dirección esta guardada como dato en la celda de memoria que identifica uniquevocamente a la dirección \textbf{0x0000}, es decir 0.
\textbf{[[0x000F]]} representa el modo de direccionamiento indirecto cuyo valor se encuentra en la celda de memoria cuya dirección esta guardada como dato en la celda de memoria que identifica uniquevocamente a la dirección \textbf{0x000F}, es decir 15.

\item \textbf{Registro}
Codigo de modo de direccionamiento: 100rrr 
Donde "rrr" es una cadena binaria que puede tomar los valores desde 000 hasta 111 para identificar según el numero decimal que representen los diferentes ocho registros R0..R7.

Ejemplos:
\textbf{R0} representa el modo de direccionamiento registro cuyo valor se encuentra en el registro \textbf{R0}.
\textbf{R7} representa el modo de direccionamiento registro cuyo valor se encuentra en el registro \textbf{R7}.

\item \textbf{RegistroIndirecto}
Codigo de modo de direccionamiento: 110rrr 
Donde "rrr" es una cadena binaria que puede tomar los valores desde 000 hasta 111 para identificar según el numero decimal que representen los diferentes ocho registros R0..R7.

Ejemplos:
\textbf{[R0]} representa el modo de direccionamiento registro cuyo valor se encuentra en la direccion de memoria que esta guardada como dato el registro \textbf{R0}.
\textbf{[R7]} representa el modo de direccionamiento registro cuyo valor se encuentra en la direccion de memoria que esta guardada como dato el registro \textbf{R7}.

\end{enumerate}


%%%%%%%%%%%%%%%%%%%%%%%%%%%%%%%%%%%%%%%%%%%%%%%%%%%%%%%%%%%
\subsection{Repertorio de instrucciones}

%%%%%%%%%%%%%%%%%%%%%%%%%%%%%%%%%%%%%%%%%%%%%%%%%%%%%%%%%%%
\subsubsection{Instrucciones de 2 operandos}
El formato de instruccion de dos operandos es el siguiente:

  CodOp   +  Modo destino +  Modo origen +  Destino  +   Origen
(4 bits)       (6 bits)        (6 bits)    (16 bits)    (16 bits)  

Donde el operando destino no puede ser un modo de direccionamiento Inmediato, ya que en la direccion de memoria o registro que describa dicho operando es donde se guardará el resultado de la operación.

\begin{enumerate}
\item \textbf{MUL destino, origen}
Código de operación: 0000
Esta instrucción describe la multiplicación entre los datos de los dos operandos. Esta operación es la única que cuyo resultado puede ser 32 bits, que son dos celdas de memoria en hexadecimal, por lo que los primeros 16 bits, es decir, la primer mitad, es guardada en el registro R7 y la segunda en el operando destino.
 
\item \textbf{ADD destino, origen}
Código de operación: 0010
Esta instrucción describe la suma entre los datos de los dos operandos. El resultado de la ejecución de la suma es guardado en el operando destino.

\item \textbf{SUB destino, origen}
Código de operación: 0011
Esta instrucción describe la resta entre los datos de los dos operandos. El resultado de la ejecución de dicha resta es guardado en el operando destino.

\item \textbf{DIV destino, origen}
Código de operación: 0111
Esta instrucción describe la división entre el dato en el operando destino como dividendo y el dato en el operando origen como divisor. El resultado de la ejecución de la división es guardado en el operando destino.

\item \textbf{MOV destino, origen}
Código de operación: 0001
Esta instrucción describe la copia de datos del dato alojado en el operando origen al operando destino. El resultado de la ejecución del MOV es el dato guardado en el operando origen ahora también guardado en el operando destino.

\item \textbf{AND destino, origen}
Código de operación: 0100
Esta instrucción describe la operación logica "y" bit a bit entre los datos de los dos operandos. El resultado de la ejecución de esta operación es guardado en el operando destino.

\item \textbf{CMP destino, origen}
Código de operación: 0110
Esta instrucción describe la comparación bit a bit entre los datos de los dos operandos. El resultado de esta operación es solamente la actualización de flags en la cpu.

\item \textbf{OR destino, origen}
Código de operación: 0101
Esta instrucción describe la operación logica "o" bit a bit entre los datos de los dos operandos. El resultado de la ejecución de esta operación es guardado en el operando destino.
\end{enumerate}

%%%%%%%%%%%%%%%%%%%%%%%%%%%%%%%%%%%%%%%%%%%%%%%%%%%%%%%%%%%
\subsubsection{Instrucciones de 1 operando origen}

El formato de instruccion de un operando origen es el siguiente:

  CodOp   +   relleno   +  Modo origen +   Origen
(4 bits)      (000000)        (6 bits)    (16 bits)

\begin{enumerate}
\item \textbf{CALL origen}
Código de operación: 1011
El efecto del CALL es guardar la direccion de memoria en la celda de la dirección que se encuentra guardadad en el SP (Stack pointer) aumentar el SP y guardar en el PC (Program Counter) el dato que se encuentra guardado en el operando origen ya que describe el llamado a una subrutina que comienza en la celda de memoria cuya direccion esta guardada en el operando origen.

\item \textbf{JMP origen}
Código de operación: 0110
El efecto del JMP es cambiar el PC (Program Counter) por el dato que esta guardado en el operando origen ya que esta operación describe el salto a otra parte de la memoria para continuar con la ejecución del programa.
\end{enumerate}

%%%%%%%%%%%%%%%%%%%%%%%%%%%%%%%%%%%%%%%%%%%%%%%%%%%%%%%%%%%
\subsubsection{Instrucciones de 1 operando destino}

El formato de instrucción de un operando destino es el siguiente:

  CodOp   +  Modo origen  +  relleno  +  Origen
(4 bits)      (6 bits)      (000000)    (16 bits)

\begin{enumerate}
\item \textbf{NOT destino}
Código de operación: 1001
Esta instrucción describe la operación logica "negación" bit a bit en el datos del operando destino. El resultado de la ejecución de esta operación es guardado en la misma celda o registro de donde es leído el dato inicialmente.
\end{enumerate}

%%%%%%%%%%%%%%%%%%%%%%%%%%%%%%%%%%%%%%%%%%%%%%%%%%%%%%%%%%%
\subsubsection{Instrucciones sin operandos}

El formato de instrucción sin operandos es el siguiente:

 CodOp     +    relleno 
(4 bits)     (000000000000)


\begin{enumerate}
\item \textbf{RET}
Código de operación: 0110
El efecto del ret es cambiar el pc por el dato que esta guardado en la celda de memoria que se encuentra en el SP (Stack pointer) y decrementar el SP ya que describe la finalización de la ejecución de una subrutina y la ejecución del resto del programa.
\end{enumerate}


%%%%%%%%%%%%%%%%%%%%%%%%%%%%%%%%%%%%%%%%%%%%%%%%%%%%%%%%%%%
\subsubsection{Instrucciones de salto condicional}

El formato de instrucción de salto condicional es el siguiente es el siguiente:

 prefijo +   CodOp   +  desplazamiento 
 (1111)     (4 bits)     (8 bits)

El efecto de cualquier salto condicional es aumentar el PC (Program Counter) en la cantidad de celdas que indique el desplazamiento si sólo la condición que cada salto condicional tiene da como resultado 1, lo cual es interpretado como verdadero.

\begin{enumerate}
\item \textbf{JE desplazamiento}
Código de operación: 0001
La condición del salto es que el flag \textbf{Z} (Cero) sea 1, es decir la ultima operación matemática dió como resultado el número cero.

\item \textbf{JNE desplazamiento}
Código de operación: 1001
La condición del salto es que el flag \textbf{Z} (Cero) sea 0, es decir la ultima operación matemática no dió como resultado el número cero.

\item \textbf{JLE desplazamiento}
Código de operación: 0010
La condición del salto es el resultado de la siguiente operación lógica \textbf{Z OR ( N XOR V )}, es decir la ultima operación matemática es menor o igual con signo.

\item \textbf{JG desplazamiento}
Código de operación: 1010
La condición del salto es el resultado de la siguiente operación lógica \textbf{NOT (Z OR ( N XOR V ))}, es decir la ultima operación matemática es mayor con signo.

\item \textbf{JL desplazamiento}
Código de operación: 0011
La condición del salto es el resultado de la siguiente operación lógica \textbf{N XOR V}, es decir la ultima operación matemática es menor con signo.

\item \textbf{JGE desplazamiento}
Código de operación: 1011
La condición del salto es el resultado de la siguiente operación lógica \textbf{NOT (N XOR V)}, es decir la ultima operación matemática es mayor o igual con signo.

\item \textbf{JLEU desplazamiento}
Código de operación: 0100
La condición del salto es el resultado de la siguiente operación lógica \textbf{C OR Z}, es decir la ultima operación matemática es menor o igual sin signo.

\item \textbf{JGU desplazamiento}
Código de operación: 1100
La condición del salto es el resultado de la siguiente operación lógica \textbf{NOT (C OR Z)}, es decir la ultima operación matemática es mayor sin signo.

\item \textbf{JCS desplazamiento}
Código de operación: 0101
La condición del salto es que el flag \textbf{C} sea 1, es decir la ultima operación matemática es menor sin signo.

\item \textbf{JNEG desplazamiento}
Código de operación: 0101
La condición del salto es que el flag \textbf{N} sea 1, es decir si el último resultado de una operación dio negativo.

\item \textbf{JVS desplazamiento}
Código de operación: 0111
La condición del salto es que el flag \textbf{V} sea 1, es decir si el último resultado de una operación dio overflow.

\end{enumerate}


%%%%%%%%%%%%%%%%%%%%%%%%%%%%%%%%%%%%%%%%%%%%%%%%%%%%%%%%%%%
\subsection{Versiones de la arquitectura (SUSI)}
hacer un grafico copado donde se vea las capas de cebolla

Las versiones de la arquitectura Q están pensadas para incorporar funcionalidades de manera que la curva de aprendizaje sea adecuada para los alumnos, siendo paulatina e incremental, es decir, cada arquitectura Qi agrega más funcionalidad (ya sean instrucciones nuevas o modos de direccionamiento) a las arquitecturas Qi anteriores.

\ojo{Incorporar lo que sigue a un gráfico de la estructura en cebolla}

%%%%%%%%%%%%%%%%%%%%%%%%%%%%%%%%%%%%%%%%%%%%%%%%%%%%%%%%%%%
\subsubsection{Q1}


\textbf{Modos de direccionamiento}
\begin{enumerate}
\item Inmediato
\item Registro
\end{enumerate}

\textbf{Instrucciones}
\begin{enumerate}
\item MOV
\item SUB 
\item DIV 
\item ADD 
\item MUL
\end{enumerate}


%%%%%%%%%%%%%%%%%%%%%%%%%%%%%%%%%%%%%%%%%%%%%%%%%%%%%%%%%%%
\subsubsection{Q2}

\textbf{Modos de direccionamiento}
\begin{enumerate}
\item Modos de direccionamiento \textbf{Q1}
\item Directo 
\end{enumerate}

\textbf{Instrucciones}
\begin{enumerate}
\item Instrucciones \textbf{Q1}
\end{enumerate}

%%%%%%%%%%%%%%%%%%%%%%%%%%%%%%%%%%%%%%%%%%%%%%%%%%%%%%%%%%%
\subsubsection{Q3}

\textbf{Modos de direccionamiento}
\begin{enumerate}
\item Modos de direccionamiento \textbf{Q2}
\end{enumerate}

\textbf{Instrucciones}
\begin{enumerate}
\item Instrucciones \textbf{Q2}
\item CALL
\item RET
\end{enumerate}

%%%%%%%%%%%%%%%%%%%%%%%%%%%%%%%%%%%%%%%%%%%%%%%%%%%%%%%%%%%
\subsubsection{Q4}

\textbf{Modos de direccionamiento}
\begin{enumerate}
\item Modos de direccionamiento \textbf{Q3}
\end{enumerate}

\textbf{Instrucciones}
\begin{enumerate}
\item Instrucciones \textbf{Q3}
\item CMP
\item JMP
\item JE 
\item JNE 
\item JLE 
\item JG 
\item JL 
\item JGE 
\item JLEU 
\item JGU 
\item JCS 
\item JNEG 
\item JVS
\end{enumerate}

%%%%%%%%%%%%%%%%%%%%%%%%%%%%%%%%%%%%%%%%%%%%%%%%%%%%%%%%%%%
\subsubsection{Q5}

\textbf{Modos de direccionamiento}
\begin{enumerate}
\item Modos de direccionamiento \textbf{Q4}
\item Idirecto
\item RegistroIdirecto
\end{enumerate}

\textbf{Instrucciones}
\begin{enumerate}
\item Instrucciones \textbf{Q4}
\end{enumerate}

%%%%%%%%%%%%%%%%%%%%%%%%%%%%%%%%%%%%%%%%%%%%%%%%%%%%%%%%%%%
\subsubsection{Q6}

\textbf{Modos de direccionamiento}
\begin{enumerate}
\item Modos de direccionamiento \textbf{Q5}
\end{enumerate}

\textbf{Instrucciones}
\begin{enumerate}
\item Instrucciones \textbf{Q5}
\item AND
\item OR
\item NOT
\end{enumerate}


%%%%%%%%%%%%%%%%%%%%%%%%%%%%%%%%%%%%%%%%%%%%%%%%%%%%%%%%%%%
\section{Estado del arte}

En un principio la materia Organizacion y Arquitectura eligio el simulador intel 8085 para enseniar assembler.

\begin{itemize}

\item \textbf{Simulador intel 8085}

Es un simulador y ensamblador para el microprocesador Intel 8085, que ofrece una interfas que a simple vista es dificil de entender, una parte para la ejecucion y otra parte para ver el estado de la memoria y registros. 
Las opiniones y experiencias con este simulador dentro de la materia no fueron muy buenas, fuera del gran esfuerzo que pusieron los profesores en que aprendieramos, dandonos mini tutoriales no resultaron como se esperaba. (Hmm si mi memoria no me falla se utilizo en un cuatrimestre [Mara si me equivoco corregime ehh!]). 
Todo lo que se trabajaba en ese simulador tambien se escribia a en hoja. 

\item En los primeros cuatrimeste la materia eligio incorporar la \textbf{Arquitectura QARQ} con su especificacion y ademas un lenguaje micro --(no me acuerdo el nombre). 
En estos tiempos se desistio en utilizar un simulador complejo pero no dejaron de buscar e investigar alguna alternativa que pueda convivir con los estudantes. Todos los ejercicios se hacian en papel, la unica forma de verificar resultados es atraves de los profesores.

\item Hace poco la materia o mejor dicho el equipo de Orga decidio dividir la especificacion de la arquitectura QARQ en varias partes, donde cada una recibe el nombre de \textbf{Arquitectura Q}. 

Cada Arquitectura Qi agrega una nueva funcionalidad (instrucciones o modos de direccionamientos) que la anterior, son como capas de cebolla. Ademas incorporaron un trabajo practico donde tiene que programar un programa que monocromatice una image en la \textbf{Arquitectura IA - 32}. En el enunciado se explica bastante como funciona esta arquitectura. Sin embargo todos los ejercicios de las arquitecturas Qi se siguen haciendo en papel.  

\item \textbf{Hasta ahora!!!} 
La idea es incorporar el \textbf{Simulador Qsim} en la materia para que los chicos puedan entender el ciclo de instruccion en la etapa de ejecucion de un programa, poder ver que es lo que estan haciendo, chequear el efecto de los programas creados, jugar y divertirse aprendiendo :P (lo siento estoy re quemada). Pensamos que nuestro simulador va a encagar en la materia ya que cubre las necesidades basicas de un simulador. :D

\end{itemize}
\part{Simulador \qsim}


\section{Funcionalidad del simulador}

La funcionalidad del simulador puede caracterizarse mediante las siguientes partes importantes:

\begin{itemize}
\item Chequeo de sintaxis de los programas escritos en el lenguaje Q
\item Ensamblado del código fuente de un programa en su correspondiente código máquina
\item Cargado en memoria del código máquina
\item Ejecución paso a paso de un programa cargado en memoria
\end{itemize}

\subsection{Chequeo de sintaxis}

El simulador provee al alumno de un editor de texto en el cual escribirá el programa en un lenguaje Qi, que desea cargar en memoria y ejecutar.
Una vez que el usuario haya terminado la escritura, al momento de cargar el programa, el simulador utilizará un parser para detectar errores de sintaxis, tales como la falta de una coma o un corchete, o la presencia de símbolos que no pertenecen al lenguaje (como por ejemplo signos de pregunta y símbolos matemáticos); o bien errores semánticos como la combinación incorrecta de elementos del lenguaje, por ejemplo: modos de direccionamiento mal ubicados.
El parser solo revisará lo escrito por el alumno y de acuerdo a las gramática del lenguaje, mostrará alguno de los siguientes estados:

\begin{description}
\item[OK] Este mensaje se obtiene cuando no hubo ningún error de sintaxis. Si se da este resultado, es posible continuar con el ensamblado y cargado en memoria.
\item[SyntaxError] Este mensaje de error se obtiene cuando en alguna línea del programa se detectó algún error de sintaxis o de semántica, como se describió arriba. Cuando ocurre este error se lo acompaña con una descripción lo mas detallada posible para que el alumno detecte donde ocurrió y pueda corregirlo. Un programa con errores no puede ser ensamblado y cargado en memoria.
\end{description}

\subsection{Ensamblado}

Una vez que el programa es sintácticamente válido es posible traducir el código fuente del programa en código máquina (representado en cadenas binarias). Para esto se respeta un formato de instrucción que indica cómo se codifica cada operación y los operandos. 

\ojo{Mas detalle al respecto de este proceso en la sección de ...}{Poner vínculo a donde se ponen ejemplos de Qi}

\subsection{Cargado en memoria}

Una vez ensamblado, la representación binaria (o código máquina) del programa será cargado en memoria a partir de una ubicación (celda de memoria) que el alumno puede elegir. Esto permite visualizar el contenido de la memoria (con el programa cargado) y el estado de los registros de la CPU. La decodificación con desensamblado permite al alumno experimentar otros escenarios y efectos laterares, entre los cuales podemos enumerar:

\begin{itemize}
\item Si la ejecución paso a paso excede los límites del programa, pueden tomarse instrucciones de otra rutina y procesarse como una nueva instruccion.
\item Si en cambio, se intenta ejecutar el contenido de una celda con datos (y no una instrucción) podrá ocurrir que se encuentre una instrucción invalida (combinacion de modos, codigos, incorrecta) y el alumno verá el mensaje de error pertinente.
\end{itemize}

Durante la carga del programa en memoria puede producirse un OutOfMemoryError?, que significa que el programa no entra en la ubicación elegida en memoria ya que ocupa más celdas que las que se encuentran disponibles debajo de ella ya que como se mencionó en la sección \ref{caracteristicasQ}, la memoria disponible tiene un tamaño limitado y por este motivo la alocación en memoria del código máquina puede exceder el espacio disponible a partir de la celda inicial anteriormente elegida. Si por el contrario, no se produce este error, el alumno podrá ver el programa cargado en memoria exitosamente.

\subsection{Ejecución paso a paso}

Se provee la funcionalidad de la ejecución paso a paso ya que se desea que el alumno pueda experimentar y así comprender los pasos del ciclo de ejecución. Además puede ejercitarse situaciones que se denominan ''errores conceptuales de programación'' Esto es a lo que llamamos Errores conceptuales, entre los cuales es posible mencionar:

\begin{itemize}
\item Tomar un dato de un sector de memoria equivocado.
\item Que el programa sobrescriba su mismo código máquina.
\item Permitir que la ejecución continue una vez terminado el programar cargado en memoria.
\item 
\end{itemize}

El paso a paso que provee el simulador consiste en las siguientes estapas pertenecientes al ciclo de instruccion:

\begin{enumerate}
\item \textbf{Búsqueda de instrucción:} El alumno podrá visualizar el valor que contiene PC (Program counter) donde se encuentra la dirección de la celda en memoria que contiene la próxima instrucción a ejecutar (por ejemplo, en caso de ser la primer instrucción del programa recién cargado, el pc tendrá la dirección de memoria elegida por el alumno para iniciar el cargado del programa en memoria). El simulador, toma de la memoria el código maquina correspondiente a la instrucción que comienza en esa dirección tomada de PC (una instrucción puede ocupar más de una celda de memoria) y los guarda en el \IR\ (\textit{Instruction Register}). Será observable también para el alumno el incremento del registro PC, tantas como celdas ocupe la instrucción actual, lo que conceptualmente es, preparar el contexto de ejecución para tomar la siguiente instrucción.

\item  \textbf{Decodificación:}
En la decodificación el Interprete se encarga de desensamblar el código máquina (abreviado en hexadecimal) que ya fue ubicado en el \IR para mostrar el código fuente de la instrucción actual con sus respectivos operandos. Si el programa escrito por el alumno es sintacticamente y conceptualmente correcto, este paso le permite comprobar que la instrucción actual es la que él mismo escribió y no otra, visualizandola en pantalla. En esta etapa se provee también la oportunidad de que el alumno aprecie otros conceptos, tales como los errores conceptuales mencionados antes.

\item  \textbf{Ejecución}
El execute ejecuta los efectos de la instrucción y muestra en pantalla los cambios en el estado de ejecución: memoria, puertos, registros y flags. Dentro de esta misma etapa se lleva a cabo el almacenamiento de resultados que, cuando sea necesario, guardará el valor resultante de la operación descripta por la instrucción en el operando destino. Esto cambiará el valor de una celda de memoria o de un registro y será visto en pantalla por el alumno.
\end{enumerate}


%%%%%%%%%%%%%%%%%%%%%%%%%%%%%%%%%%%%%%%%%%%%%%%%%%%%%%%%%%%5
\section{Implementación}

%%%%%%%%%%%%%%%%%%%%%%%%%%%%%%%%%%%%%%%%%%%%%%%%%%%%%%%%%%%
\subsection{Tecnología utilizada (tati)}

\begin{itemize}


\item  \textbf{Scala}
Elegimos el lenguaje Scala para realizar el simulador ya que, en la materia llamada Objetos III, lo utilizamos un pequeño lapso de tiempo y creímos que era una buena oportunidad para, en vez de elegir un lenguaje que hayamos utilizado más en la carrera como java, profundizar en la utilización de Scala y aprovechar las ventajas que este ofrecia al combinar el manejo de objetos y las características de un lenguaje funcional.

\item  \textbf{Arena}
Utilizamos el framework Arena para realizar la interfaz de usuario del simulador porque es un framawork de codigo abierto, que también pudimos utilizar en una de las materias y al poder ser convinado con Scala nos pareció una buena oportunidad de explotar lo que nos ofrecia para que este simulador en su totalidad sea de codigo abierto.

\item  \textbf{Eclipse}
Se eligio utilizar el entorno de programación Eclipse ya que ambas trabajamos en diferentes sistemas operativos para los cuales Eclipse es funcional y puede tener los mismos pluggins que hace que pueda sopotar proyectos MVN y Scala, además de que, al ser el entorno de programación que más hemos usado nos sentiamos cómodas con esa elección.

\item  \textbf{Git}
Elegimos git como repositorio externo para sincronizar nuestro codigo ya que promueve el codigo abierto (ACA QUIERO PONER LO DE QUE ESTA AHI ARRIBA Y NO LE CEDEMOS DERECHOS NI NADA  A NADIE PERO NO ME SALE!!!)

\end{itemize}

%%%%%%%%%%%%%%%%%%%%%%%%%%%%%%%%%%%%%%%%%%%%%%%%%%%%%%%%%%%
\subsection{ (arq OO)}

Como se observa en la figura \ref{ALU} tiente toda la responsabilidad en la ejecución de operaciones matematicas y logicas, además del analisis sobre los flags luego de cada operación. 

\graf{ALU}{Diagrama de clase de la \ALU}
%%%%%%%%%%%%%%%%%%%%%%%%%%%%%%%%%%%%%%%%

Como se observa en la figura \ref{BusEntradaSalida_Memoria_CeldasPuertos} el Bus de entrada y salida tiene la responsabilidad de derivar según donde corresponda (Memoria o Puertos) la modificación de una celda o el leer un dato. Para conoce a una instancia de la clase Memoria y a otra de la clase CeldasPuertos. 
Ambas clases conocen muchas intancias de la clase Celda, y cada Celda a su vez conoce un dato: una instancia del W16, que representa a los datos guardados en memoria o en los puertos.  

\graf{BusEntradaSalida_Memoria_CeldasPuertos}{Diagrama de clase del la \BusEntradaSalida_Memoria_CeldasPuertos}
%%%%%%%%%%%%%%%%%%%%%%%%%%%%%%%%%%%%%%%%

Como se observa en la figura \ref{CPU}, la CPU conoce a la ALU, contiene los registros IR y PC, y los flags (V,Z,C,N) y los ocho registros (R0...R7). La responsabilidad de la CPU es actualizar los flags, los registros, actualizar el PC y el IR, y ser la conexión con la ALU.

\graf{CPU}{Diagrama de clase de la \CPU}
%%%%%%%%%%%%%%%%%%%%%%%%%%%%%%%%%%%%%%%%

Como se observa en la figura \ref{Interprete} el Interprete tiene la entera responsabilidad de recibir los datos leídos desde, una celda de memoria hasta tres celdas, y bit a bit, según los códigos de operación y de modo de direccionamiento saber de que instrucción se trata y cuales son sus operandos si es que los tiene y devolver el objeto que la representa. Se ocupa del decode de la instrucción.

\graf{Interprete}{Diagrama de clase de la \Interprete}
%%%%%%%%%%%%%%%%%%%%%%%%%%%%%%%%%%%%%%%%

Como se observa en la figura \ref{ModoDireccionamiento} los modos de direccionamiento extienden del trait ModoDireccionamiento, donde se encuentran declarados mensajes necesarios para majear todas las subclases poliformicamente. Entre los cuales se encuentran los mensajes: 
\begin{itemize}
\item  \textbf{representacionString}
Que devuelve la representación en string como codigo fuente, por ejemplo la representación de un ADD(R0,R7), sería: ADD R0, R7
\item  \textbf{codigo}
Retorna el string que representa al código único de modo de direccionamiento, por el ejemplo, el código de modo de direccionamiento del R7 es 100111.
\item  \textbf{getValorString}
Retorna el string que representa al dato que posee el modo de direccionamiento. En el caso de un Inmediato que sea FF56, devolvera el string "FF56", y en el caso de cualquier registro, retornara el valor que represente su W16.
\end{itemize}
Los modos de direccionamiento diferentes a Inmediato y Registro, conocen en vez de un W16 otro modo de direccionamiento según corresponda:
\begin{itemize}
\item RegistroIndirecto conoce un a instancia de Registro.
\item Directo conoce conoce un a instancia de Inmediato.
\item Indirecto conoce conoce un a instancia de Directo.
\end{itemize}
Esto se implemento de esta manera para que el leer datos de memoria, puertos o registros, o guardarlos en los mismos sea más sencillo ya que se delega en el modo de direccionamiento que conoce.

La clase Etiqueta representa la etiqueta creada por el alumno cuando realiza el programa. Cuando el mismo es cargado en memoria, según cual sea la celda de inicio y cuanto ocupen las instruciones, se calcula la dirección de memoria que representa la etiqueta y luego se descarta reemplazandola por un modo de direccionamiento Inmediato.

La clase W16 que también esta en la figura \ref{ModoDireccionamiento}, representa el dato que es guardado en memoria. Tiene la responsabilidad de incrementarse, decrementarse, sumar una entero, devolver su representación binria y su valor en entero.


\graf{ModoDireccionamiento}{Diagrama de clase de la \ModoDireccionamiento}
%%%%%%%%%%%%%%%%%%%%%%%%%%%%%%%%%%%%%%%%

Como se observa en la figura \ref{ALU}

\graf{ALU}{Diagrama de clase de la \ALU}
%%%%%%%%%%%%%%%%%%%%%%%%%%%%%%%%%%%%%%%%

Como se observa en la figura \ref{ALU}

\graf{ALU}{Diagrama de clase de la \ALU}
%%%%%%%%%%%%%%%%%%%%%%%%%%%%%%%%%%%%%%%%

Como se observa en la figura \ref{ALU}

\graf{ALU}{Diagrama de clase de la \ALU}
%%%%%%%%%%%%%%%%%%%%%%%%%%%%%%%%%%%%%%%%

Como se observa en la figura \ref{ALU}

\graf{ALU}{Diagrama de clase de la \ALU}
%%%%%%%%%%%%%%%%%%%%%%%%%%%%%%%%%%%%%%%%

Como se observa en la figura \ref{ALU}

\graf{ALU}{Diagrama de clase de la \ALU}
%%%%%%%%%%%%%%%%%%%%%%%%%%%%%%%%%%%%%%%%


Como se observa en la figura \ref{ALU}

\graf{ALU}{Diagrama de clase de la \ALU}

%[width=0.7\textwidth]




\part{Evaluación del desarrollo}

%%%%%%%%%%%%%%%%%%%%%%%%%%%%%%%%%%%%%%%%%%%%%%%55
\section{Dificultades encontradas}

\subsection{Dificultades presentadas por el dominio}
Las dificultades del dominio estuvieron relacionadas a la comprensión no solamente del modelo de arquitectura Q si no también a su propósito didáctico, ya que el objetivo del simulador no es solamente la simulación de la arquitectura y la ejecución de programas, sino también el proveer a los alumnos la capacidad de ejercitar situaciones conceptualmente erróneas, por lo que se requería una comprensión didáctica del problema más allá de la especificación de las arquitecturas Q. 

\subsection{Dificultades de diseño}
En primera instancia se optó por implementar un modelo de objetos que utilizaba un objeto de la clase \textbf{Programa} a lo largo de toda la ejecución.Esto implica que no leía de la memoria principal las instrucciones a ejecutar, sino que solicitaba la siguiente instrucción al objeto instancia de la clase Programa, sobreviviendo así las distintas etapas una vez que fue creado por el \textit{parser}. Este enfoque evitaba la creación de un objeto que tuviera que interpretar el código máquina alojado en la memoria, \verde{evitando }{evitaba} la nueva creación de instancias de la clase \textbf{Instrucción} y \verde{haciendo mucho más sencillo al}{simplificaba en gran medida el} modelo ya que la memoria era sólo una clase que contenía datos que se reflejaban en pantalla y no se extraían datos de celdas en ningún momento. Luego entendimos que un programa no sólo podía modificar su entorno al ser ejecutado (otras celdas de memoria que no ocupen su código maquina, celdas de puertos, registros, etc) si no que también podría sobrescribir su código maquina (ya sea con ese propósito o sólo por un error conceptual), o bien, que el alumno debía tener la posibilidad de seguir ejecutando más allá del código máquina alojado en memoria o más. A partir de eso fue necesario corregir gran parte del modelo agregando una clase denominada \textbf{Intérprete}, cuya responsabilidad es interpretar la siguiente instrucción alojada en memoria para que luego sea ejecutada, otorgando más responsabilidad a la clase \textbf{Memoria} y descartando el objeto instancia de \textbf{Programa} una vez que éste es cargado en memoria con éxito.\\

Por otro lado, tuvieron que solicitarse extensiones al equipo de desarrolladores de Arena para poder realizar la interfaz con dicho framework:

\begin{itemize}
\item FileSelector para que los alumnos puedan cargar el archivo .Qsim en el cual se encuentra su programa
\item CodeEditor (O actualmente llamado KeywordTextArea). Es el espacio donde se visualiza el programa Qi que realizaron los alumnos una vez que esta cargado.
\item Bindings contra el background de componentes y celdas de una tabla, para permitir la visualización de cambios de colores en la memoria luego de las etapas de ejecución y de realizar algún cambio en la misma.
\item TexBox multiLine (TextArea) con scroll para ser utilizado como consola de devolucion.
\item Icono para la aplicación, solo por cuestiones estéticas.
\end{itemize}

%%%%%%%%%%%%%%%%%%%%%%%%%%%%%%%%%%%%%%%%%%%%%%%55
\section{Casos de prueba (revisar redaccion!!)}

\ojo{En esta sección se describen los casos de prueba de la aplicación.}{Falta explicar porqué es importante esta sección}

\subsection{Chequeo de sintaxis en la distintas Qi}

Para chequear la sintaxis de cada \textbf{Arquitectura Q}, primero escribimos un programa Qi en un archivo con extension .qsim luego este archivo lo recibe el objeto Parser que se encargara de chequear la sintaxis. 
Para eso realizamos dos casos de prueba por cada Qi:
\begin{enumerate}
\item \textbf{Chequear programa Qi valido}

\ojo{Agarramos el programa Qi válido\footnote{Un programa Qi valido es un programa en cualquier sintáticamente correcto en alguna arquitectura Q.} y se lo pasamos al Parser, cuando termina de chequear devuelve un resultado, como es de esperar devuelve un objeto programa con la lista de instrucciones. Tomamos ese resultado y como ultimo paso comparamos si el programa resultado es igual al programa esperado.}{OJO con la redacción}


\item \textbf{Chequear programa Qi es invalido} 

En este caso escribimos un programa Qi con errores de sintaxis para ver como reacciona el parser.

Agarramos el programa Qi invalido y se lo pasamos al Parser, cuando termina de chequear devuelve un error, como es de esperar devuelve una Exception \textbf{SyntaxErrorException} porque dentro del archivo del programa hay una linea que tiene errores de sintaxis, el mensaje de la excepción te muestra el numero de linea y la propia linea para que veas en donde te equivocaste. Capturamos la Excepción y tomamos como resultado el mensaje y como ultimo paso comparamos el mensaje resultado con el mensaje esperado.
\end{enumerate}

\subsection{Ensamblado}
\ojo{(Objetos --+ Código Maquina)}{}
Para poder verificar que el Ensamblado se realizo correctamente tomamos el programa Qi mencionado anteriormente y como sabemos que el resultado del parser nos devuelve un objeto de la clase \textbf{Programa} que contiene un conjunto de objetos de la clase \textbf{Instruccion}, el ensamblado simplemente lo realiza cada instrucción, esto quiere decir que cada instrucción tiene el comportamiento para generar su propio \codmaq. De esta forma realizamos el ensamblado del programa y obtenemos una lista de código maquina como resultado y como ultimo paso comparamos la lista de código maquina resultado con la lista código maquina esperado.

\subsection{Decodificación}
\ojo{Código Maquina --+ Objetos (Interprete)}{}

La Decodificación se verifica correctamente a la inversa del Ensamblado tomando el ensamblado del programa Qi que como sabemos es la lista de código maquina, la cual se itera para pasar cada elemento al objeto Interprete que es el encargado de Decodificar. Como resultado puede devolver dos cosas: 
\begin{enumerate}
\item \textbf{Lista de de Instrucciones (decodificación)} 

El resultado correcto es la lista de objetos instrucciones, por cada código maquina el interprete verifica los códigos de operación para poder crear las instrucciones correctas y como ultimo paso de la verificación comparamos la lista resultante con la lista esperada.
\item\textbf{Error}

El resultado puede ser un error porque puede pasar que el código de operación o el código de algún modo de direccionamiento dentro del formato de cada instrucción sea invalido. La excepción que puede tirar tiene el nombre de \textbf{CodigoInvalidoException}.   
\end{enumerate}

\subsection{Ejecución}

Para verificar que la ejecución de un programa Qi con el \textbf{Ciclo de Instrucción} se realice correctamente lo que hacemos es cargar el programa en la memoria a través del objeto \textbf{Simulador}, de esta forma cuando se encuentra cargado esta listo para la ejecución. Los pasos que se verifican son 3:

\begin{itemize}
\item \textbf{Fetch} 

Para verificar el fetch se toma la instrucción siguiente a ejecutar y se compara el valor que se guarda en el registro ir con el valor esperado. Ademas se verifica que el registro especial pc apunte a la siguiente instrucción a ejecutar.

\item \textbf{Decode}

Para verificar el decode, se toma el valor del registro ir para que lo reciba el interprete que dará como resultado el objeto instrucción correspondiente. Este objeto instrucción sabe mostrarse quien es y por eso la responsabilidad de decodificar se la delega a la instrucción. Al obtener la decodificación de cada instrucción podemos comparar la decodificacion resultante con la esperada.
 
\item \textbf{Execute}

Para verificar el efecto de cada instrucción se considero armar un caso de prueba por cada instrucción donde cada uno tiene un estado inicial y luego de realizar la ejecución de dicha instrucción termina con un estado final modificado en el cual verificamos el resultado obtenido con el resultado esperado. 
\end{itemize}

\textbf{Las siguiente secciones son pasos que se realizan de acuerdo al efecto de la instrucción}
  
\subsubsection{Operaciones de ALU}

La ALU es la encargada de ejecutar operaciones aritméticas/lógicas. Si la instrucción a ejecutar tiene un efecto aritmético/lógico se le delega la ejecución de la operación. Para poder verificar todas las operaciones creamos un caso de prueba por cada operación. Tanto las operaciones aritméticas como las lógicas se analizan de la misma forma:

\begin{enumerate}
\item Se toma cada valor de los operandos(la búsqueda se detalla el la siguiente instrucción)

\item Se realiza la operación aritmética/lógica. 

\item Se verifica el resultado obtenido con el resultado esperado.

\item Algunas operaciones (la mayoría) modifican los flags. la ALU tiene mucha relación con los flags ya que por cada cuenta realizada actualiza los mismos. Estos también se se verifican comparando el resultado obtenido con el valor esperado. 
\end{enumerate}

\subsubsection{Búsqueda de Operandos}

Para verificar la búsqueda de operandos, tomamos una instrucción cualquiera y probamos todas las combinaciones de modos de direccionamiento para cada operando, teniendo en cuenta que son invalidas todas las combinaciones de 1/2 operando/s que tenga el operando destino el modo de direccionamiento Inmediato. Al tener la instrucción elegida procedemos a obtener el valor de cada operando, este valor se lo compara con el valor esperado. 
 
\subsubsection{Almacenamiento de Operandos}

Para verificar el almacenamiento de operandos, salvo algunas instrucciones todas pasan por la etapa de store.
Tomamos cada una de las instrucciones de un programa Qi y realizamos el ciclo de ejecución mencionado anteriormente.
Cuando estamos en la etapa de ejecución, cada una tiene un operando destinado a guardar el resultado de su efecto.
En esta instancia sabemos que  dicho operando tiene el valor guardado, a este valor lo comparamos con el valor esperado.

%%%%%%%%%%%%%%%%%%%%%%%%%%%%%%%%%%%%%%%%%%%%%%%55
\section{Ejemplos de uso (Revisar!!)}

En esta sección vamos a mostrar las diferentes situaciones que pueden ocurrir a la hora de utilizar el simulador.

\subsection{Para Experimentar}

Esta sección describe situaciones particulares que los alumnos pueden experimentar.

\begin{itemize}

\item Supongamos que un alumno escribe el siguiente programa Qi:

1.ADD R0, [0x0002]\\
2.MUL R4, 0x0001 \\
3.SUB [0x0003], 0x000A \\
4.MOV R5, 0x0056 \\
5.MOV [0x0005], etiqueta \\
6.CALL [0x0005] \\
7.etiqueta: ADD R0, 0x0002 \\
8.RET \\

Lo interesante es que luego de ejecutar el CALL se ejecuta la instrucción ADD R0, 0x0002 y el RET. Cuando vuelve para ejecutar la próxima instrucción el PC se encuentra en la linea 7, donde vuelve a ejecutar la instrucción ADD antes mencionada. Ya lo ultimo que le queda es ejecutar el RET.
Vamos a hacer un mapa del estado del registro SP antes de seguir: Antes de ejecutar el RET  el registro SP tiene el valor inicial que es FFEF, luego en la ejecución lo primero que se hace es incrementar el SP osea que ahora tiene el valor FFFO y luego buscar el valor de esa dirección para actualizar el registro PC. Para informarles la dirección FFFO es un puerto de E/S. La conclusión es que en esta instancia el flujo de ejecución del programa depende del valor que tiene ese puerto, puede pasar que el valor sea 0x0000 y se actualice el PC nos lleve al inicio de la memoria donde se encuentra inicializado otro programa y empiece a ejecutar desde allí. 


\item Para los que son curiosos, supongamos que un alumno escribe el siguiente programa Qi:
\begin{verbatim}
ADD R0, [0x0002]
MUL R4, 0x0001
SUB [0x0003], 0x000A 
\end{verbatim}

Pensemos que el PC se posiciona en la linea 3 y realizamos el ciclo de instrucción(FETCH - DECODE - EXECUTE). Como estado final tenemos el efecto de la ultima instrucción y el valor de PC apuntando a la siguiente instrucción a ejecutar. Como verán el programa que escribió el alumno termino en la la linea 3 pero el simulador no para de ejecutar, por ende te permite realizar las veces que quiera el ciclo de instrucción. Es re interesante que los chicos experimenten este tipo de cosas.

\end{itemize} 

\verde{La decodificación con desensamblado permite al alumno experimentar otros escenarios y efectos laterales, entre los cuales podemos enumerar:

\begin{itemize}
\item Si la ejecución paso a paso excede los límites del programa, pueden tomarse instrucciones de otra rutina y procesarse como una nueva instruccion.
\item Si en cambio, se intenta ejecutar el contenido de una celda con datos (y no una instrucción) podrá ocurrir que se encuentre una instrucción invalida (por ejemplo, una combinación incorrecta de modos de direccionamiento y códigos de operación) y el alumno verá el mensaje de error pertinente.
\end{itemize}}{ACOMODAR!!!}


%%%%%%%%%%%%%%%%%%%%%%%%%%%%%%%%%%%%%%%%%%%%%%%%%%%
\section{Trabajo Futuro}

En esta sección se describirán características y funcionalidades que deseamos agregar al Simulador QSim en el futuro.

\begin{itemize}

\item Habilitar las instrucciones PUSH y POP.

Estas instrucciones permiten el manejo de la Pila como estructura de datos disponible para el programador. 
PUSH tiene como efecto agregar el valor del operando origen a la pila. 
POP permite sacar el primer elemento de la pila y guardarlo en el operando destino.
Actualmente estas instrcucciones se encuentran implementadas pero no habilitadas en ninguna gramática Qi.

\item Entrada y Salida.

Que el simulador admita la interacción con dispositivos de E/S. Para eso tenemos que modelar dispositivos como el teclado, impresora, monitor, etc.

\end{itemize}




%%%%%%%%%%%%%%%%%%%%%%%%%%%%%%%%%%%%%%%%%%%%%%%%%%%%
\begin{thebibliography}{9}

\bibitem{Stallings} Williams Stallings, \emph{Computer Organization and Architecture}, octava edición, Editorial Prentice Hall, 2010.

\bibitem{Tanenbaum} A. Tanenbaum, \emph{Organización de Computadoras}, cuarta edición, Editorial Pearson.

\bibitem{Patterson} Hennessy, Patterson. \emph{Arquitectura de Computadores - Un enfoque cuantitativo}, primera edición,  Editorial Mc Graw Hill.

\bibitem{blog} Sitio oficial de la materia \orga: \url{http:\\orga.blog.unq.edu.ar} (2013)




\end{thebibliography}

\end{document}